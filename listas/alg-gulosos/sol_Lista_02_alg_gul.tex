\documentclass[12pt, a4paper]{article}
\title{Solução Lista 02 - Algoritmos Gulosos - CANA 2025.1}
\author{Cornélio A. de S.}
\date{Maio 2025}

\usepackage[portuguese]{babel}
\usepackage{amsmath}
\usepackage{amssymb}
\usepackage{dirtytalk}
\usepackage{listings}

\usepackage{geometry}
\geometry{top=2cm, left=2cm, right=2cm}

\lstset{
  frame=tb,
  language=Python,
  aboveskip=3mm,
  % belowskip=3mm,
  showstringspaces=false,
  % columns=flexible,
  basicstyle={\small\ttfamily},
  % numbers=none,
  % numberstyle=\tiny\color{gray},
  % keywordstyle=\color{blue},
  % commentstyle=\color{dkgreen},
  % stringstyle=\color{mauve},
  % breaklines=true,
  % breakatwhitespace=true,
  tabsize=4
}

\begin{document}


\maketitle


\section*{Q1}

\subsection*{(a)}

Desenvolvendo o somatório do tamanho da codificação, temos:

\[
\sum_{i=1}^{n} mp_i \log_{2}{\frac{1}{p_i}}
= \sum_{i=1}^{n} mp_i \log_{2}{2^{k_i}}
= \sum_{i=1}^{n} mp_i k_i
\]

Percebe-se que a última equação é a fórmula para o custo de uma codificação de Huffman, onde $k_i$ representa o custo de um símbolo de frequência $mp_i$, ou seja, sua profundidade na árvore de codificação. Portanto, para provarmos a equação acima, basta mostrar que um símbolo terá profundidade $k$ se e somente se possuir probabilidade $1/2^k$. Faremos essa prova por indução, começando com o caso base que são os símbolos com profundidade $1$. Porém, antes disso, vamos provar que, dada as restrições do problema (probabilidades da forma $1/2^{k_i}$), os filhos de um mesmo nó devem sempre ter a mesma probabilidade, pois essa prova nos será útil durante a indução:

\textbf{Propriedade do parentesco:} filhos de um mesmo nó devem ter probabilidades iguais. Consequentemente, cada filho tem metade da probabilidade do pai.

Vamos provar a propriedade acima. Digamos que os símbolos a serem codificados são $\{ a_1, a_2, \ldots, a_n \}$ (em ordem crescente de probabilidade) e que o símbolo de menor frequência, $a_1$, tenha probabilidade $1/2^k$. Demostraremos que, para que a soma das probabilidades de todos os símbolos seja igual a $1$, é necessário que o símbolo $a_2$ também tenha probabilidade $1/2^k$.

Dado que $1/2^k$ é a menor probabilidade dentre os símbolos do alfabeto, podemos ter no máximo $2^k$ símbolos com probabilidade $1/2^k$, uma vez que $2^k \cdot 1/2^k = 1$. Se o alfabeto tiver menos que $2^k$ símbolos ($n < 2^k$), podemos somar as probabilidades $1/2^k$ entre si de modo a reduzir o número de probabilidades e manter o somatória igual a $1$. Porém, pra obtermos apenas símbolos com probabilidades da forma $1/2^{k_i}$, temos que obrigatoriamente combinar duas probabilidades iguais:

\[ \frac{1}{2^j} + \frac{1}{2^j} = 2 \cdot \frac{1}{2^j} = \frac{1}{2^{j-1}} \]

Dentre as $2^k$ probabilidades iguais à $1/2^k$, pelo menos uma não pode ser somada as outras, uma vez que assumimos que pelo menos o símbolo $a_1$ tem probabilidade igual à $1/2^k$. Nos resta então $2^k - 1$ probabilidades iguais à $1/2^k$ para serem combinadas entre si duas à duas de modo a obtermos probabilidades da forma $1/2^{k-1}$. Como $2^k - 1$ é ímpar, irá sobrar pelo menos uma probabilidade $1/2^k$ que não pode ser combinada, fazendo com que pelo menos mais um símbolo tenha probabilidade $1/2^k$, neste caso o símbol $a_2$ que é o segundo com menor probabilidade dentre os símbolos do alfabeto.

Uma vez que $a_2$ também tem probabilidade $1/2^k$, o algoritmo formador da árvore mínima da codificação de Huffman irá combinar $a_1$ e $a_2$ em um novo símbolo $S$ de probabilidade $P(S)=1/2^{k-1}$. Se $P(a_3)$ também for igual a $1/2^k$, teremos a mesma situação novamente, o que irá obrigar $P(a_4)=1/2^k$ e fará com que o algoritmo de Huffman combine $a_3$ e $a_4$, e assim por diante até serem esgotados os símbolos de probabilidade $1/2^k$. Uma vez que o algoritmo de Huffman combinar todos os símbolos de probabilidade $1/2^k$, teremos um novo alfabeto $\{ a'_1, a'_2, \ldots, a'_{n-p} \}$ (onde $p$ é o número de combinações que foram feitas) cujas probabilidades também são potências de $2$ e cuja menor potência corresponde ao símbolo $a'_1$ com o valor de $1/2^{k-1}$. As mesmas ideias podem ser aplicadas à esse alfabeto, o que obriga $a'_2$ a também ter probabilidade $1/2^{k-1}$ e faz com que o algoritmo de Huffman combine $a'_1$ e $a'_2$. Portanto, por indução, o algoritmo de Huffman irá sempre combinar símbolos de probabilidades iguais, o que significa que todo os nós com filhos terão filhos com a mesma probabilidade. $\blacksquare$

Agora, vamos provar por indução que os símbolos de profundidade $k$ possuem probabilidade $1/2^k$ e vice-versa, começando com o caso base que é $k=1$.

\textbf{CASO BASE:} Nós (nós internos ou folhas) de profundidade $1$ são filhos do nó raiz. Ambos os filhos da raiz devem ter probabilidades iguais (segundo a propriedade do parentesco acima) e uma vez que o nó raiz tem probabilidade $1$ (a raiz acumula a probabilidade de todos os símbolos) os seus dois filhos devem ter probabilidade $1/2$. Portanto, símbolos de profundidade $1$ devem ter probabilidade $1/2$. Ademais, não é possível que símbolos com probabilidade $1/2$ tenham profundidade maior que $1$, uma vez que todo nó abaixo da profundidade $1$ deve ter probabilidade menor que $1/2$, visto que são descendentes de nós cuja probabilidade é $1/2$.

\textbf{HIPÓTESE INDUTIVA:} Para $k \in \{1, 2, \ldots, h \}$ os nós de profundidade $k$ possuem probabilidade $1/2^k$ e vice-versa.

\textbf{PASSO INDUTIVO:} Dada como verdadeira a hipótese indutiva, temos que provar que os símbolos de profundidade $h+1$ possuem probabilidade $1/2^{h+1}$. Para um símbolo ter profundidade $h+1$ ele tem que ser filho de um nó de profundidade $h$. Nós de profundidade $h$, pela hipótese indutiva, tem probabilidade $1/2^h$ e seus filhos, segundo a propriedade do parentesco, devem ter probabilidades iguais. Portanto, nós de profundidade $h+1$ devem ter probabilidade $1/2^{h} \div 2 = 1/2^{h+1}$. Ademais, não é possível que símbolos com probabilidade $1/2^{h+1}$ tenham profundidade maior que $h+1$, uma vez que todo nó abaixo da profundidade $h+1$ deve ter probabilidade menor que $1/2^{h+1}$.

Portanto, segue por indução que símbolos cuja probabilidade são da forma $1/2^{k_i}$ possuem profundidade $k$ na árvore de codificação mínima de Huffman se e somente se tiverem probabilidade $1/2^k$. Essa propriedade é equivalente a equação de tamanho da codificação:

\[ \sum_{i=1}^{n} mp_i \log_{2}{\frac{1}{p_i}} \]

\subsection*{(b)}

A fórmula é máxima quando a distribuição é uniforme. A fórmula é mínima quando apenas um valor concentra toda a probabilidade, ou seja, tem probabilidade $1$ enquanto os outros valores tem prababilidade $0$.


\section*{Q2}

Modificar o algoritmo de Kruskal para obter a Árvore Geradora Máxima em vez da mínima (inverter os pesos ou ordenar em reverso) e retornar as arestas em $E$ que não fazem parte desta árvora. Solução no arquivo \textit{/src/cana/feedback.py}.


\section*{Q3}

\subsection*{(a)}

Para um símbolo ter uma codificação de comprimento $1$, este símbolo deve \say{sobreviver} até a última combinação de símbolos que forma a raíz da árvore. Consideremos que os símbolos em ordem crescente de frequência sejam $\{a_1, a_2, \ldots, a_{n-1}, a_n\}$, onde $F(a_i)=f_i$ é a frequência do i-ésimo símbolo $a_i$ e tendo $f_n > 2/5$. Podemos ter no máximo 2 símbolos com frequência maior ou igual a $2/5$, então, se $f_{n-1}$ também for maior (ou igual a) que $2/5$, temos que:

\[ \left( \sum_{i=0}^{n-2} f_i \right) < 1/5\]

Dessa forma, o algoritmo de Huffman irá combinar todos os símbolos do índice $1$ até o índice $n-2$ em um \say{novo} símbolo, vamos chamá-lo de $S$, cuja frequência é $F(S) = \sum_{i=0}^{n-2} f_i < 1/5$. Dessa forma, nos restará 3 símbolos $S$, $a_{n-1}$ e $a_n$, com frequências $F(S) < f_{n-1} \leq f_n$. O restante do processo de formação da árvore mínima irá combinar $S$ e $a_{n-1}$ e o resultado será combinado com $a_n$, fazendo com que $a_n$ tenha uma codificação de comprimento unitário.

Agora vamos cobrir o caso em que $a_{n-1} < 2/5$. Considere o momento em que o processo de combinação de símbolos do algoritmo de Huffman resulte \emph{no primeiro símbolo com frequência maior ou igual a $f_n$}, de modo que tenhamos agora os símbolos $\{S_1, S_2, \ldots, S_{j-1}, S_j\}$ em ordem crescente de frequência. Como foi a primeira vez que o algoritmo retornou uma combinação de símbolos com frequência maior (ou igual a) que $f_n$, essa combinação deve ser o símbolo $S_j$, que é agora o símbolo com a maior frequência dentre as combinações. Ademais, temos que ter obrigatoriamente $S_{j-1}=a_n$, pois se $a_n$ já tiver sido combinado com algum outro símbolo, $S_j$ não seria o primeiro a ter frequência maior que $f_n$, o que é um absurdo frente a nossa consideração inicial.

Obviamente, se $f_n \geq 1/2$, não é possível que uma combinação de símbolos que não contenha $a_n$ tenha uma frequência maior que $f_n$, mas nesses casos é óbvio que a codificação de $a_n$ terá comprimento unitário, uma vez que o algoritmo de Huffman iria combinar todos os símbolos $\{a_1, a_2, \ldots, a_{n-1} \}$ para só então combinar o resultado com $a_n$.

Agora temos que provar por absurdo que os símbolos $\{ S_1, S_2, \ldots, S_{j-2} \}$ não podem existir. Temos as seguintes inequações até agora:

\[ F(S_{j-1}) = F(a_n) = f_n > 2/5 \]
\[ \Rightarrow F(S_j) \geq f_n > 2/5 \]
\[ \Rightarrow F(S_{j-1}) + F(S_j) > 4/5 \]
\[ \Rightarrow \left( \sum_{i=1}^{j-2} F(S_i) \right) < 1/5 \]

Agora vamos nos questionar sobre os símbolos $K_1$ e $K_2$ que formaram $S_j$:

\[ F(K_1) + F(K_2) = F(S_j) \geq f_n > 2/5 \]

Para que a frequência de $S_j$ seja maior que $2/5$, um dos símbolos $K_1$ ou $K_2$ deve ter uma frequência maior que $(2/5 \div 2)=1/5$. Porém, isso é um absurdo, pois o algoritmo de Huffman apenas combina os símbolos de menor frequência dentre todos os que estão disponíveis e nós temos os símbolos $\{ S_1, S_2, \ldots, S_{j-2} \}$ todos com frequências menores que $1/5$. A única forma de solucionar este absurdo é se $\{ S_1, S_2, \ldots, S_{j-2} \}$ não existirem, havendo apenas $\{a_n, S_j\}$. Dessa forma, $a_n$ também terá codificação unitária quando $F(a_{n-1})=f_{n-1}<2/5$. $\blacksquare$

\subsection*{(b)}

Novamente, para um símbolo ter uma codificação de comprimento $1$, este símbolo deve \say{sobreviver} até a última combinação de símbolos que forma a raíz da árvore. Vamos supor que existe um conjunto de símbolos $\{a_1, a_2, \ldots,a_n\}$ com frequências $\{f_1, f_2, \ldots, f_n\}$ todas menores que $1/3$, ordenados de forma crescente com base nas frequências. Para haver uma codificação de comprimento $1$ é necessário que a sequência de combinações de símbolos do algoritmo de Huffman resulte no seguinte conjunto de símbolos:

\[ \{a_n, S\} \quad \textrm{onde} \quad F(a_n) < 1/3 \quad \textrm{e} \quad F(S) > 2/3 \]

Digamos que $S$ foi formado pela combinação dos símbolos $K_1$ e $K_2$. Para tal, esses símbolos devem ser os dois símbolos com as menores frequências no conjunto $\{K_1, K_2, a_n\}$. Como $F(S)>2/3$, um dos símbolos $K_1$ ou $K_2$ deve ter frequência maior que $1/3$, o que é um absurdo, pois $K_1$ e $K_2$ devem ter frequências menores ou iguais à frequência do símbolo $a_n$. $\blacksquare$


\section*{Q4}

Vamos assumir que $G$ é conectado. A estratégia é remover todos os vértices em $U$ do grafo $G$, aplicar o algoritmo de Kruskal para determinar a árvore geradora mínima neste novo grafo reduzido e depois adicionar os nós removidos usando as arestas incidentes de menor custo para cada um.

Se qualquer um dos vértices em $U$ for um nó essencial para a conectividade do grafo, ou seja, ao remover o nó o grafo que antes era totalmente conectado se torna 2 ou mais componentes conectadas, não é possível que tal nó seja uma folha em qualquer que seja a árvore geradora construída. É possível identificar tais situações analisando o resultado do algorítmo de Kruskal: se o número de aresta retornado por Kruskal for menor que $\lvert V - U \rvert - 1$, então pelo menos um dos vértices removidos era essencial para a conectividade do grafo e o resultado final não é alcançável.


\section*{Q5}

Basta utilizar a mesma estrutura de dados de árvores enraizadas para conjuntos disjuntos utilizada no algoritmo de Kruskal. Utilizamos a união por rank para unir árvores com base no conjunto de restrições de igualdade. Uma vez feita todas as uniões e esgotado os pares de igualdades, basta verificar a validade das restrições de desigualdade: para uma desigualdade ser válida, as variáveis da desigualdade devem estar em árvores enraizadas distintas, caso contrário, a desigualdade é inválida e as restrições não são satisfatível.


\section*{Q6}

Atender os clientes em ordem crescente do tempo para serem servidos. Ou seja, começar com os clientes que podem ser atendidos mais rapidamente. Algoritmos de ordenação eficientes tem complexidade $O(n\log{n})$. Se os tempos forem pequenos ou se forem discretos, a ordenação pode ser feita em $O(n)$. Essa estratégia é ótima uma vez que o tempo de espera total é dado pela fórmula:

\[ T = n \cdot t_1 + (n-1) \cdot t_2 + \ldots + (n-k+1) \cdot t_k + \ldots + 2 \cdot t_{n-1} + 1 \cdot t_n \]

Onde $t_1$ é o tempo de serviço do primeiro cliente atendido, $t_k$ é o tempo de serviço do k-ésimo cliente atendido e $t_n$ é o tempo de serviço do último cliente atendido. Como os clientes que são atendidos primeiro possuem um multiplicador maior, para minimizar a soma basta colocar os clientes que são atendidos mais rapidamente no começo da fila.

\end{document}
